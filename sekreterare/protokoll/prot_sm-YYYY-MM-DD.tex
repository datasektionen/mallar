% Mall för SM-protokoll
% 
% Jesper Särnesjö, sarnesjo@kth.se, 2007-06-30
% Modifierad av Cecilia Roes, roes@kth.se, 2013-02-02
% 
% Följande sök-och-ersätt-variabler finns:
% 
% @ordf                 Mötesordförandens namn
% @sekr                 Mötessekreterarens namn
% @justA , @justB        Justeringspersonernas namn
% @startTid, @slutTid   Tidpunkterna för mötets öppnande respektive avslutande, på formen 12:34.

\documentclass{protokoll}

\usepackage[swedish]{babel}
\usepackage[T1]{fontenc}
\usepackage[utf8]{inputenc}
\usepackage[final]{pdfpages}
\bild{dsekt}
\typ{Typ-SM} % Byt ut Typ
\datum{YYYY-MM-DD} % Fyll i datum
\organ{Sektionsmötet}
\organisation{Konglig Datasektionen}
\ordf{@ordf}
\sekr{@sekr}
\justA{@justA}
\justB{@justB}

\begin{document}

{\small \textit{Sektionsmöten (SM) är hålls minst fyra gånger per år där alla sektionsmedlemmar välkomnas att deltaga för att besluta i sektionsövergripande frågor. Sådana kan exempelvis gälla val av funktionärer, ändringar i sektionens rambudget och i styrdokumenten. SM anordnas av sektionsstyrelsen, D-rektoratet, och därför kontaktar du med fördel denna på \textbf{\textit{drek@d.kth.se}} vid eventuella frågor.}} % ändra text då det känns lämpligt

\begin{motesfakta} % Fyll i tid, plats och närvarande.
  \tid{@starttid -- @sluttid}
  \plats{plats, KTH}
  \narvarande
    \person{}{\textit{se bilaga}} % antingen att man fyller i alla personer eller hänvisar till inscannad bilaga.
%    \person{Namn, D-XX}{Post}
\end{motesfakta}


\punkt{Mötets högtidliga öppnande}
%  @ordf förklarade mötet öppnat @starttid.


%\punkt{Presentation och utfrågning av urnvalskandidater}


\punkt{Formalia}
  \begin{enumerate}
    \item\textbf{Val av mötesordförande}
%      @ordf valdes till mötesordförande.
    \item\textbf{Val av mötessekreterare}
%      @sekr valdes till mötessekreterare.
    \item\textbf{Val av två justeringspersoner tillika rösträknare}
%      @justA och @justB valdes till justeringspersoner tillika rösträknare.
    \item\textbf{Mötets behöriga utlysande}
%      Mötet förklarades behörigt utlyst.
%    \item\textbf{Mötesordförandes reglementesenliga klädedräkt} %Endast under Glögg-SM
    \item\textbf{Eventuella adjungeringar}
%      Inga adjungeringar.
%      X, Y och Z gavs yttranderätt.
    \item\textbf{Anmälan av övriga frågor}
%      Inga övriga frågor.
%      \begin{enumerate}
%        \item \textbf{X anmäler frågan ""}
%      \end{enumerate}
    \item\textbf{Fastställande av föredragningslistan}
%      X yrkade för att
%      Föredragningslistan fastställdes % med ovanstående ändringar.
    \item\textbf{Tidigare mötens protokoll}
%      Protokoll för xx-SM ÅÅÅÅ-MM-DD lades till handlingarna.
  \end{enumerate}


\punkt{Rapporter}

Se bilaga för mer information.

  \begin{enumerate}
    \item \textbf{D-rektoratet}
      \begin{enumerate}
        \item \textbf{Presidiet}
        \item \textbf{Kassör}
        \item \textbf{Sekreterare}
        \item \textbf{Ledamot för sociala frågor och relationer}
        \item \textbf{Ledamot för utbildningsfrågor}
        \item \textbf{Ledamot för studiemiljöfrågor}
      \end{enumerate}
    \item \textbf{Ekonomisk lägesrapport}
    \item \textbf{Förvaltningsrevisionsrapport}
    \item \textbf{Övriga funktionärer}
      \begin{enumerate}
      	\item \textbf{Chefredaqtör}
      	\item \textbf{Datas Art Director}
        \item \textbf{Ärkedemon}
        \item \textbf{DESCtop}
        \item \textbf{D-Dagenansvarig}
        \item \textbf{Internationella Utskottets ordförande}
        \item \textbf{Klubbmästare}
        \item \textbf{Fanbärare}
        \item \textbf{Jämlikhetsnämndens ordförande}
        \item \textbf{Kårfullmäktigeledamöter}
        \item \textbf{Kommunikatör}
        \item \textbf{Konglig Lokalchef}
        \item \textbf{Ljud- och ljusansvarig}
        \item \textbf{Mottagningsansvariga}
        \item \textbf{Mulle/Mullerina Schmeck}
        \item \textbf{Näringslivsansvarig}
        \item \textbf{Programansvarig student}
        \item \textbf{Prylmånglaren}
        \item \textbf{Qulturattaché}
        \item \textbf{Revisorer}
        \item \textbf{Sektionshistoriker}
        \item \textbf{Sektionsidrottsledare}
        \item \textbf{Studerande skyddsombud}
        \item \textbf{Studienämndens ordförande}
        \item \textbf{Systemansvarig}
        \item \textbf{Valberedningens ordförande} 
        \item \textbf{Öfvermatrona}
      \end{enumerate}
    \item \textbf{Kåren}
    \item \textbf{Projekt}
      \begin{enumerate}
%       Inkludera projektledare för årets projekt
        \item \textbf{Projektledare för ... 20XX}
      \end{enumerate}
    \item \textbf{Sektionen för Medieteknik}
    \item \textbf{Skolan för datavetenskap och kommunikation (CSC)}
  \end{enumerate}


%\punkt{Bordlagda ärenden}
%  \begin{enumerate}
%    \item \textbf{...}
%  \end{enumerate}


%\punkt{Andra läsningen}
%  \begin{enumerate}
%    \item \textbf{Stadgeändring angående }
%      \begin{beslut}
%        \att bifalla/avslå propositionen om stadgeändring.
%      \end{beslut}
%  \end{enumerate}


%\punkt{Revisionsärenden}
%  \begin{enumerate}
%    \item \textbf{Verksamhetsberättelse för 20XX}
%      \begin{beslut}
%        \att
%      \end{beslut}

%    \item \textbf{Ekonomisk berättelse för 20XX}
%      \begin{beslut}
%        \att
%      \end{beslut}

%    \item \textbf{Revisionsberättelse och fråga om ansvarsfrihet för verksamhetsåret 20XX}
%      \begin{beslut}
%        \att
%      \end{beslut}
%  \end{enumerate}


%\punkt{Beslutsärenden}
%  \begin{enumerate}
%    \item \textbf{...}
%      \begin{beslut}
%        \att
%      \end{beslut}
%  \end{enumerate}


%\punkt{Interpellationer}
%  \begin{enumerate}
%    \item \textbf{Interpellation av }	
%  \end{enumerate}


%\punkt{Propositioner}
%  \begin{enumerate}
%    \item \textbf{Proposition angående }
%      X föredrog propositionen.
%      \begin{beslut}
%        \att bifalla/avslå propositionen.
%      \end{beslut}
%  \end{enumerate}


%\punkt{Motioner}
%  \begin{enumerate}
%    \item\textbf{Motion angående }
%      X föredrog motionen.
%      Y föredrog motionssvaret.
%      \begin{beslut}
%        \att bifalla/avslå motionen.
%      \end{beslut}	
%  \end{enumerate}


%\punkt{Valärenden}
%  \begin{enumerate}
%    \item \textbf{Val av X}
%      Y presenterade sig.
%      \begin{beslut}
%        \att välja Y till X.
%      \end{beslut}
%  \end{enumerate}


%\punkt{Hedersmedlem}


%\punkt{Hedersdelta}


\punkt{Övriga frågor}
%  \begin{enumerate}
%    \item \textbf{}
%  \end{enumerate}


\punkt{Nästa möte}
%  Annonseras vid ett senare tillfälle.


\punkt{Mötets högtidliga avslutande}
%  ordf förklarade mötet avslutat kl. @sluttid


\punkt*{Bilagor}
  \begin{enumerate}
    \item\textbf{Närvarolista}
    \item\textbf{In- och utlista}
    \item\textbf{Rapporter}
%    \item\textbf{Proposition angående }
%    \item\textbf{Motion angående }
%    \item\textbf{Motionssvar angående }
%    \item\textbf{Ändringsyrkande/yrkande angående }
  \end{enumerate}

% För att kunna inkluder pdf-filer, t.ex proppar, motioner m.m
% \includepdf[pages=-]{./xxx/xx.pdf}

% Inkuldera pdf i liggande format
% \includepdf[pages=-, landscape]{./xxx/xx.pdf}

\end{document}
