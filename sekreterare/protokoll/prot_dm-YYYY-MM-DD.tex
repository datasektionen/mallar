% Mall för DM-protokoll
% 
% Jesper Särnesjö, sarnesjo@kth.se, 2007-06-30
% Modifierad av Cecilia Roes, roes@kth.se, 2013-02-02
% 
% Följande sök-och-ersätt-variabler finns:
% 
% @ordf                 Mötesordförandens namn
% @sekr                 Mötessekreterarens namn
% @justA                Justeringspersonens namn
% @startTid, @slutTid   Tidpunkterna för mötets öppnande respektive avslutande, på formen 12:34.

\documentclass{protokoll}

\usepackage[swedish]{babel}
\usepackage[T1]{fontenc}
\usepackage[utf8]{inputenc}
\usepackage[final]{pdfpages}
\bild{dsekt}
\typ{Styrelsen}
\datum{YYYY-MM-DD} % Fyll i datum
\organ{DM}
\organisation{Konglig Datasektionen}
\ordf{@ordf}
\sekr{@sekr}
\justA{@justA}

\begin{document}

{\small \textit{D-Rektoratsmöten (DM) är styrelsemöten som hålls minst två gånger per termin. På dessa informerar, diskuterar, fastslår och beslutar Datasektionens styrelse kring motionssvar, vissa avtal, medel från dispositionsfonden, ledamöter i valberedningen, m.m. Alla medlemmar  har rätt att närvara och är hjärtligt välkomna på DM. Kontakta Sekreterare, \textit{\textbf{sekr@d.kth.se}}, för frågor om protokollet.}} % Ändra text när det känns lämpligt

\begin{motesfakta} % Fyll i tid, plats och närvarande.
  \tid{@starttid -- @sluttid}
  \plats{plats, KTH}
  \narvarande
    \person{}{\textit{se bilaga}} % antingen att man fyller i alla personer eller hänvisar till inscannad bilaga.
%    \person{Namn, D-XX}{Post}
\end{motesfakta}


\punkt{Mötets högtidliga öppnande}
%  @ordf förklarade mötet öppnat @starttid.


\punkt{Formalia}
  \begin{enumerate}
    \item\textbf{Val av mötesordförande}
%      @ordf valdes till mötesordförande.
    \item\textbf{Val av mötessekreterare}
%      @sekr valdes till mötessekreterare.
    \item\textbf{Val av justeringsperson tillika rösträknare}
%      @justA valdes till justeringsperon tillika rösträknare.
    \item\textbf{Mötets behöriga utlysande}
%      Mötet förklarades behörigt utlyst.
    \item\textbf{Eventuella adjungeringar}
%      Inga adjungeringar.
%      X, Y och Z gavs yttranderätt.
    \item\textbf{Anmälan av övriga frågor}
%      \begin{enumerate}
%        \item \textbf{X anmäler frågan ""}
%      \end{enumerate}
    \item\textbf{Fastställande av föredragningslistan}
%      X yrkade för att
%      Föredragningslistan fastställdes % med ovanstående ändringar.
    \item\textbf{Tidigare mötens protokoll}
%      Protokoll för xx-DM ÅÅÅÅ-MM-DD lades till handlingarna.
  \end{enumerate}


\punkt{Rapporter}
  \begin{enumerate}
    \item \textbf{D-rektoratet}
      \begin{enumerate}
        \item \textbf{Presidiet}
        \item \textbf{Kassör}
        \item \textbf{Sekreterare}
        \item \textbf{Ledamot för sociala frågor och relationer}
        \item \textbf{Ledamot för utbildningsfrågor}
        \item \textbf{Ledamot för studiemiljöfrågor}
      \end{enumerate}
    \item \textbf{Ekonomisk rapportering}
        \item \textbf{Lägesrapport}
        \item \textbf{Ändringar i sektionens detaljbudget}
  \end{enumerate}


%\punkt{Bordlagda ärenden}
%  \begin{enumerate}
%    \item \textbf{...}
%      \begin{beslut}
%        \att
%      \end{beslut}
%  \end{enumerate}


%\punkt{Beslutsärenden}
%  \begin{enumerate}
%    \item \textbf{...}
%      \begin{beslut}
%        \att
%      \end{beslut}
%  \end{enumerate}


%\punkt{Interpellationer}
%  \begin{enumerate}
%    \item \textbf{Interpellation av }
%      \begin{beslut}
%        \att lägga interpellationen.
%      \end{beslut}
%  \end{enumerate}


%\punkt{Propositioner}	
%  \begin{enumerate}
%    \item \textbf{Proposition angående }
%      \begin{beslut}
%        \att lägga propositionen.
%      \end{beslut}
%  \end{enumerate}


%\punkt{Motioner}	
%  \begin{enumerate}
%    \item \textbf{Motion angående }
%      \begin{beslut}
%        \att yrka för bifall/avslag av motionen och att X skriver motionssvar.
%      \end{beslut}	
%  \end{enumerate}


%\punkt{Valärenden}
%  \begin{enumerate}
%    \item \textbf{Val av X}
%      \begin{beslut}
%        \att välja Y till X.
%      \end{beslut}
%  \end{enumerate}


%\punkt{Hedersmedlem}


\punkt{Övriga frågor}
%  Inga övriga frågor.
%  \begin{enumerate}
%    \item \textbf{}
%  \end{enumerate}


%\punkt{Hedersdelta}


\punkt{Nästa möte}
%  Nästa möte kommer att annonseras vid ett senare tillfälle.


\punkt{Mötets högtidliga avslutande}
%  @ordf förklarade mötet avslutat @sluttid


\punkt*{Bilagor}
%  \begin{enumerate}
%    \item\textbf{Närvarolista}
%    \item\textbf{Proposition angående }
%    \item\textbf{Motion angående }
%    \item\textbf{...}
%  \end{enumerate}

% För att kunna inkluder pdf-filer, t.ex proppar, motioner m.m
% \includepdf[pages=-]{./xxx/xx.pdf}

%Inkuldera pdf i liggande format
% \includepdf[pages=-, landscape]{./xxx/xx.pdf}

\end{document}
