% Mall för DM-dagordning
% 
% Jesper Särnesjö, sarnesjo@kth.se, 2007-06-30
% Modifierad av Cecilia Roes, roes@kth.se, 2013-02-02

\documentclass{dagordning}

\usepackage[swedish]{babel}
\usepackage[T1]{fontenc}
\usepackage[utf8]{inputenc}
\usepackage[final]{pdfpages}
\bild{dsekt}
\typ{XXX-DM} % Fyll i typ av DM
\datum{YYYY-MM-DD} % Fyll i datum
\organ{DM}
\organisation{Konglig Datasektionen 802412-7709}

\begin{document}


{\small \textit{D-Rektoratsmöten (DM) är öppna styrelsemöten som hålls minst två gånger per termin. På dessa informeras det om, diskuteras, fastslås och tas beslut gällande motionssvar, vissa avtal, medel från dispositionsfonden, ledamöter i valberedningen, m.m. Alla medlemmar har rätt att närvara på DM, men endast styrelsen har rösträtt. Övrig reglering anslås i stadgarnas §4.2-3.}}\\

\punkt{Mötets högtidliga öppnande}

\punkt{Formalia}
  \begin{enumerate}
    \item \textbf{Val av mötesordförande}
    \item \textbf{Val av mötessekreterare}
    \item \textbf{Val av justeringsperson tillika rösträknare}
    \item \textbf{Mötets behöriga utlysande}
    \item \textbf{Eventuella adjungeringar}
    \item \textbf{Anmälan av övriga frågor}
    \item \textbf{Fastställande av föredragningslistan}
    \item \textbf{Tidigare mötens protokoll}
  \end{enumerate}


\punkt{Rapporter}
  \begin{enumerate}
    \item \textbf{D-rektoratets verksamhet}
	\begin{enumerate}
        \item \textbf{Presidiet}
        \item \textbf{Kassör}
        \item \textbf{Sekreterare}
        \item \textbf{Ledamot för utbildningsfrågor}
        \item \textbf{Ledamot för näringsliv och kommunikation}
        \item \textbf{Ledamot för studiesociala frågor}
	\end{enumerate}
    \item \textbf{Ekonomisk rapportering}
	\begin{enumerate}
        \item \textbf{Lägesrapportering}
        \item \textbf{Eventuella detaljbudgetändringar}
	\end{enumerate}
  \end{enumerate}


% Avkommentera de punkter du behöver.
% Om det t.ex. är flera motioner lägg till extra items under den rubriken.


%\punkt{Informationspunkter}
%  \begin{enumerate}
%    \item \textbf{...}
%  \end{enumerate}

%\punkt{Avtal som ingåtts}
%  \begin{enumerate}
%     \item \textbf{...}
% \end{enumerate}

%\punkt{Bordlagda ärenden}
%  \begin{enumerate}
%    \item \textbf{...}
%  \end{enumerate}


%\punkt{Beslutsärenden}
%  \begin{enumerate}
%    \item \textbf{...}
%  \end{enumerate}


%\punkt{Interpellationer}   % Endast vid DM innan SM
%  \begin{enumerate}
%    \item \textbf{Interpellation av}
%  \end{enumerate}


%\punkt{Propositioner}   % Endast vid DM innan SM
%  \begin{enumerate}  
%    \item \textbf{Proposition angående }
%  \end{enumerate}


%\punkt{Motioner}   % Endast vid DM innan SM
%  \begin{enumerate}
%    \item \textbf{Motion angående }
%  \end{enumerate}



%\punkt{Valärenden}
%  \begin{enumerate}
%    \item \textbf{Val av }
%  \end{enumerate}


%\punkt{Hedersmedlem}


\punkt{Övriga frågor}


%\punkt{Hedersdelta}


\punkt{Nästa möte}


\punkt{Mötets högtidliga avslutande}


\punkt*{Bilagor}
% \begin{enumerate}
%   \item\textbf{...}
% \end{enumerate}

% För att kunna inkluder pdf-filer, t.ex proppar, motioner m.m

% \includepdf[pages=fromp-top]{./xxx/xx.pdf}
 %för att kunna lägga saker i mappar, använd %\includepdf[pages=-]{./folder/namn.pdf}
 
% Inkuldera pdf i liggande format 
% \includepdf[pages=fromp-top, landscape]{./xxx/xx.pdf} 

\end{document}
