% Mall för SM-dagordning
% 
% Jesper Särnesjö, sarnesjo@kth.se, 2007-06-30
% Modifierad av Cecilia Roes, roes@kth.se, 2013-02-02

\documentclass{dagordning}

\usepackage[swedish]{babel}
\usepackage[T1]{fontenc}
\usepackage[utf8]{inputenc}
\usepackage[final]{pdfpages}
\bild{dsekt}
\typ{Typ-SM}       % Byt ut Typ
\datum{YYYY-MM-DD} % Fyll i datum
\organ{Sektionsmötet}
\organisation{Konglig Datasektionen}

\begin{document}

{\small \textit{Sektionsmöten (SM) är hålls minst fyra gånger per år där alla sektionsmedlemmar välkomnas att deltaga för att besluta i sektionsövergripande frågor. Sådana kan exempelvis gälla val av funktionärer, ändringar i sektionens rambudget och i styrdokumenten. SM anordnas av sektionsstyrelsen, D-rektoratet, och därför kontaktar du med fördel denna på \textbf{\textit{drek@d.kth.se}} vid eventuella frågor.}}\\

\punkt{Mötets högtidliga öppnande}


% Om urnval sker på mötet så borde presentation ske tidigt.
% \punkt{Presentation och utfrågning av urnvalskandidater}


\punkt{Formalia}
  \begin{enumerate}
    \item\textbf{Val av mötesordförande}
    \item\textbf{Val av mötessekreterare}
    \item\textbf{Val av två justeringspersoner tillika rösträknare}
    \item\textbf{Mötets behöriga utlysande}
    %Inför Glögg-SM avkommentera följande:
%    \item\textbf{Mötesordförandes reglementesenliga klädedräkt}
    \item\textbf{Eventuella adjungeringar}
    \item\textbf{Anmälan av övriga frågor}
    \item\textbf{Fastställande av föredragningslistan}
    \item\textbf{Tidigare mötens protokoll}
  \end{enumerate}

\punkt{Rapporter}
  \begin{enumerate}
    \item \textbf{D-rektoratet}
      \begin{enumerate}
        \item \textbf{Presidiet}
        \item \textbf{Kassör}
        \item \textbf{Sekreterare}
        \item \textbf{Ledamot för sociala frågor och relationer}
        \item \textbf{Ledamot för utbildningsfrågor}
        \item \textbf{Ledamot för studiemiljöfrågor}
      \end{enumerate}
    \item \textbf{Ekonomisk lägesrapport}
    \item \textbf{Förvaltningsrevisionsrapport}
    \item \textbf{Övriga funktionärer}
      \begin{enumerate}
      	\item \textbf{Chefredaqtör}
      	\item \textbf{Datas Art Director}
        \item \textbf{Ärkedemon}
        \item \textbf{DESCtop}
        \item \textbf{D-Dagenansvarig}
        \item \textbf{Internationella Utskottets ordförande}
        \item \textbf{Klubbmästare}
        \item \textbf{Fanbärare}
        \item \textbf{Jämlikhetsnämndens ordförande}
        \item \textbf{Kårfullmäktigeledamöter}
        \item \textbf{Kommunikatör}
        \item \textbf{Konglig Lokalchef}
        \item \textbf{Ljud- och ljusansvarig}
        \item \textbf{Mottagningsansvariga}
        \item \textbf{Mulle/Mullerina Schmeck}
        \item \textbf{Näringslivsansvarig}
        \item \textbf{Programansvarig student}
        \item \textbf{Prylmånglaren}
        \item \textbf{Qulturattaché}
        \item \textbf{Revisorer}
        \item \textbf{Sektionshistoriker}
        \item \textbf{Sektionsidrottsledare}
        \item \textbf{Studerande skyddsombud}
        \item \textbf{Studienämndens ordförande}
        \item \textbf{Systemansvarig}
        \item \textbf{Valberedningens ordförande} 
        \item \textbf{Öfvermatrona}
      \end{enumerate}
    \item \textbf{Kåren}
    \item \textbf{Projekt}
      \begin{enumerate}
%       Inkludera projektledare för årets projekt
        \item \textbf{Projektledare för ... 20XX}
      \end{enumerate}
    \item \textbf{Sektionen för Medieteknik}
    \item \textbf{Skolan för datavetenskap och kommunikation (CSC)}
  \end{enumerate}


% Avkommentera de punkter ni behöver. Lägg till fler items om de behövs.


%\punkt{Bordlagda ärenden}
%  \begin{enumerate}
%    \item \textbf{...}
%  \end{enumerate}


%\punkt{Andra läsningen}
% Lägg till alla stadgeändringar från förra SM här, de måste ha en andra läsning.
%  \begin{enumerate}
%    \item \textbf{Stadgeändring angående }  
%  \end{enumerate}


% Avkommentera följande endast för Revisions-SM.
%\punkt{Revisionsärenden}
%  \begin{enumerate}
%    \item \textbf{Verksamhetsberättelse för 20XX}
%    \item \textbf{Ekonomisk berättelse för 20XX}
%    \item \textbf{Revisionsberättelse och fråga om ansvarsfrihet för verksamhetsåret 20XX}
%  \end{enumerate}


%\punkt{Beslutsärenden}
%  \begin{enumerate}
%    \item \textbf{...}
%  \end{enumerate}


%\punkt{Interpellationer}
%  \begin{enumerate}
%    \item \textbf{Interpellation av }  
%  \end{enumerate}


%\punkt{Propositioner}
%  \begin{enumerate}
%    \item{\textbf{Proposition angående }
%  \end{enumerate}


%\punkt{Motioner}
%  \begin{enumerate}
%    \item \textbf{Motion angående }
%  \end{enumerate}


%\punkt{Valärenden}
% Bordlagda valärenden hanteras under "Bordlagda ärenden"
%  \begin{enumerate}
%    \item \textbf{Val av }
%  \end{enumerate}


%\punkt{Hedersmedlem}

% Avkommentera hedersdelta inför Revisions-SM
%\punkt{Hedersdelta}


%\punkt{Övriga frågor}
%  \begin{enumerate}
%    \item \textbf{...}
%  \end{enumerate}


\punkt{Nästa möte}


\punkt{Mötets högtidliga avslutande}


\punkt*{Bilagor}
  \begin{enumerate}
    \item\textbf{Rapporter}
%    \item\textbf{Proposition angående }
%    \item\textbf{Motion angående }
%    \item\textbf{Motionssvar angående }
  \end{enumerate}

% För att kunna inkluder pdf-filer, t.ex proppar, motioner m.m
% \includepdf[pages=-]{./xxx/xx.pdf}

% Inkuldera pdf i liggande format
% \includepdf[pages=-, landscape]{./xxx/xx.pdf}

\end{document}
