\documentclass[a4paper,11pt]{article}

\usepackage[T1]{fontenc}
\usepackage[swedish]{babel}
\usepackage[utf8]{inputenc}
\usepackage{graphicx}
\usepackage{fancyhdr}
\usepackage{url}
\usepackage[final]{pdfpages}
\pagestyle{fancy}

\newcommand{\HRule}{\rule{\linewidth}{0.5mm}}


\title{$\delta$Funk-policy}
\author{\textsc{Konglig Datasektionen}}
\date{Antagen på Glögg-SM 2013}


\begin{document}


\maketitle

\section{Skyldigheter}
För de som antar ett förtroendeuppdrag som $\delta$funktionär är det inte endast dans och lek på röda rosor. Som $\delta$funktionär har man ett ansvartsområde och det finns ett antal skyldigheter som vi i D-rektoratet vill ska följas av alla $\delta$funktionärer. Den här texten ska försöka ge klarhet i vilka dessa skyldigheter är och vad man kan få ut av att följa dem.

\subsection{Mail}
Som $\delta$funktionär har du en funktionsmailadress som D-rektoratet och andra som vill komma i kontakt med dig kommer att använda. Kontrollera din funktionärsmail och besvara den i alla fall minst var tredje dag.

\subsection{Överlämning}
När din mandatperiod tar slut tar någon annan över. Det är din skyldighet att finnas tillgänglig och kunna svara på frågor och hjälpa din efterträfare på posten, så att ingen behöver börja om från början. Det är också rekomenderat att du och din efterträdare håller kontakten även efter överlämningen så att frågor som dyker upp kan ställas och besvaras. På samma sätt bör du känna till att du kan kontakta den du tar över efter.

\subsection{Testamente}
För att ingenting ska hamna mellan stolarna mellan dig och din efterträdare, och för att samma hjul inte ska behöva uppfinnas igen, så är det din skyldighet som funktionär att skriva ett testamente när mandatperioden är slut. 

\subsection{Rapporter}
Till varje Sektionsmöte förväntas du som $\delta$funktionär skriva en rapport om vad du har gjort sedan det senaste Sektionsmötet. Detta för att sektionen ska veta vad du gör och har gjort.

\subsection{Gå på utbildning}
Vi annordnar två stycken funktionärsutbildningar per år. Om du verkligen inte kan gå på den första har du ett halvår på dig att försöka kunna gå den andra. Det är din skyldighet att gå på minst en av dessa.

\subsection{Utvecklingssamtal}
Din kontaktperson i D-rektoratet kommer minst en gång per år att hålla ett utvecklingssamtal med dig. När det är dags, är det din skyldighet att hjälpa till att hitta en tid där ett sådant kan ägarum, och sedan att gå på det. Det är bra både för dig själv, och för sektionen i stort, att ha ett utvecklingssamtal.

\subsection{Teambuildning}
Flera gånger under året kommer D-rektoratet att bjuda in till olika sorters teambuildningevent som $\delta$funktionärer kommer att bli bjuden till. Det är starkt rekommenderat att gå på dessa, då de kan ses både som ett utbildningstillfälle men också som ett viktigt tillfälle att förstärka relationerna bland dig och de andra $\delta$funktionärna. Förutom att vara roliga förstås.

\subsection{Belöningar}
För ett bra utfört arbete under din tid som $\delta$funktionär, och ifall du har gjort ovan punkter med belåtenhet så finns det ett antal saker D-rektoratet gör för att visa sin uppskattning:

Varje år på våren anordnar D-rektoratet en skiftesgasque som alla avgående och nyvalda $\delta$funktionärer som utfört ett bra arbete kommer att bli bjudna till.
En $\delta$funktionär som gjort ett bra jobb, kommer efter dennes avgång även att få en medalj från D-rektoratet för att visa vår uppskattning till denne.

\end{document}
